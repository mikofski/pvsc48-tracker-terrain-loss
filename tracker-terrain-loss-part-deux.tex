% !TEX spellcheck = en_US

\documentclass[conference]{IEEEtran}
\usepackage{cite}
\usepackage{amsmath,amssymb,amsfonts}
\usepackage{algorithmic}
\usepackage{graphicx}
\usepackage{textcomp}
\usepackage{xcolor}
% add hyperlinks, delete all .aux files if adding hyperref after previous build
\usepackage{hyperref}
% support for unicode charcters like "é" and "ñ"
\usepackage[T1]{fontenc}
% Provides generic commands \degree, \celsius, \perthousand, \micro and \ohm
\usepackage{gensymb}
% splits a section into multiple columns
\usepackage{multicol}
\def\BibTeX{{\rm B\kern-.05em{\sc i\kern-.025em b}\kern-.08em
    T\kern-.1667em\lower.7ex\hbox{E}\kern-.125emX}}
\begin{document}

\title{Tracker Terrain Loss Part Two}

\author{\IEEEauthorblockN{Mark A. Mikofski}
	\IEEEauthorblockA{DNV, Oakland, CA, 9612, USA }}

\maketitle

\begin{abstract}
Trackers on variable terrain can incur electric mismatch losses from row-to-row shading, even with backtracking. Standard and slope-aware backtracking algorithms only eliminate row-to-row shade for trackers on flat ground. Tracker terrain loss is the difference between the theoretically best performance of trackers on flat ground and the performance of trackers using standard backtracking algorithms but on variable terrain. We used SolarFarmer to study tracker terrain loss by simulating the Hopewell Friends Solar power plant, which has an average 4\% southwest slope. We calculated a tracker terrain loss of 2.4\% when the entire site was modeled as one layout for both 5-minute and 1-hour input data. By subdividing the site into two and three layouts, the tracker terrain loss decreased to 1.8\% and 1.6\% respectively. For this particular site, neither higher frequency input data nor slope-aware backtracking significantly affected the tracker terrain loss. This study is a continuation of a previous study that prompted improvements in SolarFarmers shading algorithm. The results of this study demonstrate that SolarFarmer can now be used to calculate tracker terrain loss.
\end{abstract}

\begin{IEEEkeywords}
trackers, terrain, losses
\end{IEEEkeywords}

\section{Introduction}
Write something about \ref{fig:layouts}

\begin{figure}[htbp]
\centerline{\includegraphics[width=9cm]{Hopewell_Civil_Base.png}}
\caption{Contour map of terrain at Hopewell, NC, single-axis array.}
\label{fig:hopewell_contour_map}
\end{figure}

\begin{figure}[htbp]
\centerline{\includegraphics[width=9cm]{layouts.png}}
\caption{Sketch of 3-layout model, layout \#2 with 30 strings is outlined in red, layout \#1 with 54 strings is to the west and layout \#3 with 27 strings is to the east. The 2-layout model combines layout \#2 and \#3 for 57 strings total, while layout \#1 remains the same.}
\label{fig:layouts}
\end{figure}

Write more stuff and make references to \cite{Marion2013}.

\section{Methods}

\subsection{Site Characteristics}

The study uses a single-axis array funded by the Department of Energy and built by Cypress Creek Renewables \cite{CypressCreekRenewables2019} near Asheboro, NC at a latitude and longitude of 35.627994$^\circ$ and -79.872853$^\circ$ respectively. The site is asymmetric, with 25-qty variable length rows of 2-in-portrait and 18-modules wide single-axis trackers. The modules are 1.978-meters long, and the rows are spaced about 7.8-meters apart, so the GCR is about 51.5\%. There are 18-qty Longi LR6-72BP-360 360-watt bifacial modules per string, and three strings per each Huawei SUN2000-25KTL-US 25-kW inverter. The inverters each have three inputs, so that's one string per input.

The terrain has a generally southwestern slope as shown contour map in Fig.~\ref{fig:hopewell_contour_map} with slightly steeper southern slopes on the east side of the array and milder western slopes on the north side of the array. The maximum and average slopes for each aisle from west to east in the array are summarized in Table~\ref{table:ew-slope-summary} starting with aisle 1 at the top northern tip of the array. The maximum and average north-south slopes for every 5 rows are shown in Table~\ref{table:row-slope-summary} starting with the 1st row on the western side of the array.

\begin{table}[htbp]
\caption{Summary of East-West Slopes}
\begin{center}
\begin{tabular}{|c|c|c|c|}
\hline
\textbf{Aisle} & \textbf{\textit{Maximum}}& \textbf{\textit{Average}}& \textbf{\textit{Direction}} \\
\hline
1& 7.81& 6.22& west \\
\hline
2& 7.58& 6.59& west \\
\hline
3& 6.7& 5.77& west \\
\hline
4&5.99& 5.26& west \\
\hline
5& 5.08& 4.28& west \\
\hline
6& 4.69& 3.54& west \\
\hline
7& 4.42& 3.28& west \\
\hline
8& 4.06& 3.99& west \\
\hline
\end{tabular}
\label{table:ew-slope-summary}
\end{center}
\end{table}

\begin{table}[htbp]
\caption{Summary of North-South Slopes}
\begin{center}
\begin{tabular}{|c|c|c|c|}
\hline
\textbf{Row} & \textbf{\textit{Maximum}}& \textbf{\textit{Average}}& \textbf{\textit{Direction}} \\
\hline
1&  1.76&  1.11& south \\
\hline
5&  2.04&  1.77& south \\
\hline
10& 2.95&  2.54& south \\
\hline
15& 4.81&  4.35& south \\
\hline
20& 6.66&  5.84& south \\
\hline
25& 8.35&  4.96& south \\
\hline
\end{tabular}
\label{table:row-slope-summary}
\end{center}
\end{table}

\subsection{Model Simulation}

The system was modeled using SolarFarmer \cite{Mikofski_8547323} which allows parallel trackers to be oriented in any direction on any slope either in a plane or following the terrain. The Hopewell Friends Solar array was not modeled exactly the same as built or specified. The modeled site has 111-qty trackers with a total of 222 strings for a total DC size of 1.44-MW, which is slightly smaller than the actual size of 1.45-MW. Also the modeled site has 74 inverters with a DC/AC ratio less than one to remove the effects of clipping, versus the actual site which has only 45 inverters and a DC/AC ratio of 1.3. Also the simulated trackers spacing is uniform throughout the array and the tracker aisles are aligned, which also differs from the site as built and specified. Finally the panels were treated as monofacial, because bifacial is currently only possible for 2-dimensional simulations, and for this study we needed to take advantage of 3-dimensional modeling to capture the terrain. These changes were made for ease of modeling, and to make sure that variable row spacing and other artifacts did not affect the results, since we only want to observe the effect of slope.

The site was modeled 24 times with both in-plane and follow terrain modes for trackers with a single layout, 2 layouts, and 3-layout models, see Fig.~\ref{fig:layouts}. As shown in the layout summary in Table~\ref{table:system-summary} this results in layouts with different orientations, so they will track slightly differently. The layout-plane contains all of the trackers when they are modeled in a single plane, and is the plane that determines how the trackers backtrack when they follow the terrain. Each layout was simulated with both standard \cite{Marion2013} and slope-aware \cite{Anderson2020} backtracking. Both 5-minute and hourly irradiance for the site were obtained from the NREL PSM3 website using pvlib-python.

\begin{table}[htbp]
\caption{Summary Tracker Layouts}
\begin{center}
\begin{tabular}{|c|c|c|c|}
\hline
\textbf{Number of Layouts} & \textbf{\textit{1}}& \textbf{\textit{2}}& \textbf{\textit{3}} \\
\hline
number of trackers&    111& 54&  54 \\
        &     &   57&  30 \\
        &     &     &  27 \\
\hline
layout-plane azimuth (\degree)& 236.9&  247.26&  246.78 \\
       &      &  223.03&  238.47 \\
       &      &        &  207.58 \\
\hline
layout-plane tilt (\degree)&    2.83&    2.92&    2.89 \\
    &        &    2.98&    3.58 \\
    &        &        &    3.26 \\
\hline
tracker axis tilt (\degree)&   1.55&    1.13&    1.14 \\
         &       &    2.18&    1.87 \\
         &       &        &    2.89 \\
\hline
tracker side-slope (\degree)&  2.37& 2.69&    2.66 \\
          &     & 2.03&    3.05 \\
          &     &     &    1.51 \\
\hline
\end{tabular}
\label{table:system-summary}
\end{center}
\end{table}


\section{Results}

The tracker terrain loss was calculated using the following formula:

\begin{equation}
\text{Tracker Terrain Loss} = 1 - \frac{Y_\text{terrain}}{Y_\text{plane}}\label{eq:tracker-terrain-loss}
\end{equation}

In \ref{eq:tracker-terrain-loss}, $Y_\text{terrain}$ and $Y_\text{plane}$ are the energy yield in $kWh/kW_p$ of the trackers following the terrain and in a mono-plane respectively. The simulated energy yield for 5-minute input data with standard backtracking are shown in Table~\ref{table:standard-5min}. Each row refers to two simulations in a plane and following terrain, respectively, and subdivided into either 1, 2, or 3 layouts for a total of six simulations. The tracker terrain loss can be calculated using \ref{eq:tracker-terrain-loss} as 2.4\%, 1.8\%, and 1.6\% for 1, 2, or 3 layouts respectively.

Table~\ref{table:slope-5min} repeats the simulations with slope-aware backtracking enabled. The calculated tracker terrain losses are nearly identical to the standard backtracking case. However, it's interesting that the global incident irradiance (GI) is slightly greater, by about 0.01\% for slope-aware backtracking, presumably because the trackers take advantage of the southwest slope, and the in-plane energy yield is also negligibly greater, but the following terrain simulations is slightly lower. The non-uniform shade from variable terrain has an outsized effect on performance and quickly negates any gains form slope-aware backtracking

\begin{table}[htbp]
\caption{Energy Yield for Tracker Layouts for 5-minute Input Data with Standard Backtracking}
\begin{center}
\begin{tabular}{|c|c|c|c|c|c|}
\hline
\textbf{Lay-}& \textbf{\textit{Terrain}}& \textbf{\textit{GI}}&        \textbf{\textit{Yield}}&        \textbf{\textit{Shading}}& \textbf{\textit{Mismatch}} \\
\textbf{outs}& \textbf{\textit{Mode}}&    \textbf{\textit{$kWh/m^2$}}& \textbf{\textit{$kWh / kW_p$}}& \textbf{\textit{\%}}&      \textbf{\textit{\%}} \\
\hline
1& In Plane& 2044.7&  1695.9& 2.1& 0 \\
 & Follow&         &  1655.8& 2.4& 2.1 \\
\hline
2& In Plane& 2045.9&  1696.2& 2.2& 0 \\
 & Follow&         &  1665.2& 2.3& 1.7 \\
\hline
3& In Plane& 2046.9&  1697.6& 2.2& 0 \\
 & Follow&         &  1670.8& 2.2& 1.4 \\
\hline
\end{tabular}
\label{table:standard-5min}
\end{center}
\end{table}


\begin{table}[htbp]
\caption{Energy Yield for Tracker Layouts for 5-minute Input Data with Slope-Aware Backtracking}
\begin{center}
\begin{tabular}{|c|c|c|c|c|c|}
\hline
\textbf{Lay-}& \textbf{\textit{Terrain}}& \textbf{\textit{GI}}&        \textbf{\textit{Yield}}&        \textbf{\textit{Shading}}& \textbf{\textit{Mismatch}} \\
\textbf{outs}& \textbf{\textit{Mode}}&    \textbf{\textit{$kWh/m^2$}}& \textbf{\textit{$kWh / kW_p$}}& \textbf{\textit{\%}}&      \textbf{\textit{\%}} \\
\hline
1& In Plane& 2045&    1695.9& 2.2& 0 \\
 & Follow&       &    1654.8& 2.4& 2.2 \\
\hline
2& In Plane& 2046.2&  1696.2& 2.2& 0 \\
 & Follow&         &  1664.3& 2.3& 1.7 \\
\hline
3& In Plane& 2047.2&  1697.7& 2.2& 0 \\
 & Follow&         &  1669.8& 2.3& 1.5 \\
\hline
\end{tabular}
\label{table:slope-5min}
\end{center}
\end{table}

\begin{table}[htbp]
\caption{Energy Yield for Tracker Layouts for 1-hour Input Data with Standard Backtracking}
\begin{center}
\begin{tabular}{|c|c|c|c|c|c|}
\hline
\textbf{Lay-}& \textbf{\textit{Terrain}}& \textbf{\textit{GI}}&        \textbf{\textit{Yield}}&        \textbf{\textit{Shading}}& \textbf{\textit{Mismatch}} \\
\textbf{outs}& \textbf{\textit{Mode}}&    \textbf{\textit{$kWh/m^2$}}& \textbf{\textit{$kWh / kW_p$}}& \textbf{\textit{\%}}&      \textbf{\textit{\%}} \\
\hline
1& In Plane& 2048.4&  1700.6& 2.1& 0 \\
 & Follow&         &  1659.1& 2.4& 2.2 \\
\hline
2& In Plane& 2049.9&  1700.9& 2.2& 0 \\
 & Follow&         &  1669.2& 2.3& 1.7 \\
\hline
3& In Plane& 2050.7&  1702.3& 2.2& 0 \\
 & Follow&         &  1675&   2.3& 1.5 \\
\hline
\end{tabular}
\label{table:standard-1hr}
\end{center}
\end{table}


\begin{table}[htbp]
\caption{Energy Yield for Tracker Layouts for 1-hour Input Data with Slope-Aware Backtracking}
\begin{center}
\begin{tabular}{|c|c|c|c|c|c|}
\hline
\textbf{Lay-}& \textbf{\textit{Terrain}}& \textbf{\textit{GI}}&        \textbf{\textit{Yield}}&        \textbf{\textit{Shading}}& \textbf{\textit{Mismatch}} \\
\textbf{outs}& \textbf{\textit{Mode}}&    \textbf{\textit{$kWh/m^2$}}& \textbf{\textit{$kWh / kW_p$}}& \textbf{\textit{\%}}&      \textbf{\textit{\%}} \\
\hline
1& In Plane& 2048.7&  1700.7& 2.2& 0 \\
 & Follow&         &  1658.1& 2.4& 2.2 \\
\hline
2& In Plane& 2050.2&  1701&   2.2& 0 \\
 & Follow&         &  1668.2& 2.3& 1.8 \\
\hline
3& In Plane& 2051.1&  1702.4& 2.2& 0 \\
 & Follow&         &  1674.1& 2.3& 1.5 \\
\hline
\end{tabular}
\label{table:slope-1hr}
\end{center}
\end{table}


\section{Conclusions}
The Hopewell Friends Solar power plant was simulated with SolarFarmer to calculate tracker terrain loss. This site has variable terrain and an average 4\% southwest slope. The tracker terrain loss for the entire site was 2.4\%. When dividing the site into two and three layouts, the tracker terrain loss decreased to 1.8\% and 1.6\% respectively. The simulations were repeated with 5-minute and 1-hour input data, but the tracker terrain loss was insensitive to input data resolution for this particular site. The simulations were also repeated with slope-aware backtracking algorithm, but it had no effect on the tracker terrain loss. Future work would be to use SolarFarmer's independent tracker position time-series input with advanced tracking algorithms to measure the recovery.

\section*{Acknowledgment}

Data and layout specifications for the Hopewell, NC solar array were provided by PV Evolution Labs and Cypress Creek Renewables based on funding by the U.S. Department of Energy, Energy Efficiency and Renewable Energy as detailed in the funding opportunity announcement, DE-FOA-0001840 \cite{CypressCreekRenewables2019}.

\bibliographystyle{IEEEtran}
% argument is your BibTeX string definitions and bibliography database(s)
\bibliography{IEEEabrv,bibliography}

\end{document}
